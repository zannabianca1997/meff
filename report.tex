\documentclass[italian]{article}
\usepackage[T1]{fontenc}
\usepackage[utf8]{inputenc}
\usepackage{lmodern}
\usepackage[a4paper]{geometry}
\usepackage{babel}
\usepackage{amsmath}
\usepackage{amssymb}
\usepackage{mathtools}

\newcommand{\approxusing}[1]{\stackrel{#1}{\approx}}

\newcommand{\lagrangian}{\mathcal{L}}
\newcommand{\action}{\mathcal{S}}

\newcommand{\ox}{\hat{x}}
\newcommand{\oy}{\hat{y}}
\newcommand{\meff}{m_{\mathrm{eff}}}

\newcommand{\tc}{\quad\mathrm{t.c.}\quad}

\newcommand{\deriv}[2]{\frac{\partial #1}{\partial #2}}
\newcommand{\diff}{\mathrm{d}}

\title{Massa effettiva di una particella in un potenziale periodico}
\author{Giuseppe Zanichelli}

\begin{document}
    Come è noto dalla teoria delle bande di Bloch, una particella sottoposta a un potenziale periodico presenta uno spettro duplice: una parte continua, caratterizzata da energia maggiore di zero, e diverse bande formate da stati legati, ognuna centrata intorno ad un'autovalore dell'hamiltoniana di singola buca e con larghezza determinata dall'interazione tra stati centrati in buche vicine.
    Nel limite $T\to0$ un'espansione al termine quadratico della densità di stati della banda più bassa ci mostra che la particella può essere considerata libera, ma con una massa effettiva diversa dall'originale.
    \begin{align}
        \lagrangian = \frac{m}{2}\left(\deriv{\ox}{t}\right)^2 - V(\ox) \approxusing{T \to 0} \frac{\meff}{2}\left(\deriv{\ox}{t}\right)^2 - c && V(x+l) = V(x) 
    \end{align}
    Uno studio analitico di $\meff$ è impossibile se non nei casi più semplici. Buoni risultati si ottengono con approsimazioni di \textit{tight binding} ma lasciando parametri da misurare. 
    
\section{Azione, adimensionalizzazione e discretizzazione} 

    L'azione euclidea adimensionalizzata del sistema su un percorso $x(\tau)$, con $V(x)$ di periodo $l$ è:
    \begin{equation}
        \action_E = \frac{1}{\hbar} \int_0^{\frac{\hbar}{k_B T}} \diff\tau \frac{m}{2}\left(\deriv{x}{\tau}\right)^2 + V(x)
    \end{equation}
    Volendo studiare le dinamiche del reticolo adimensionalizziamo l'equazione tenendo le dimensioni di esso in conto. Usando $V_0$ come parametro di profondità delle buche e $l$ come misura di lunghezza, possiamo fare le seguenti sostituzioni:
    \begin{align*}\label{eq:adimensional_subs}
        x &= ly  & V(x)&=V_0 \tilde{V}\left(\frac{x}{l}\right)  & \omega_B &= \sqrt{\frac{V_0}{m l^2}} \\
        z &= \omega_B \tau & \tilde{\beta} &= \frac{\hbar \omega_B}{k_B T} & d &
        = \frac{V_0 l^2 m}{\hbar^2}
    \end{align*}
    L'azione ora è funzione solo dei parametri adimensionali $\tilde{\beta}$ e $d$ (e del cammino $y(z)$).
    \begin{equation}
        \action_E = \sqrt{d} \int_0^{\tilde{\beta}} \diff z \frac{1}{2} \left(\deriv{y}{z}\right)^2 + \tilde{V}\left(y\right)
    \end{equation}
    Infine possiamo discretizzare l'integrale dividendolo in parti di lunghezza $\eta$, ponendo $N = \frac{\tilde{\beta}}{\eta}$.
    \begin{equation}
        \action_E \approx \sqrt{d} \sum_{n=0}^N \frac{y_n^2}{\eta} - \frac{y_n y_{n+1}}{\eta} + \eta \tilde{V} \left(y_n\right) 
    \end{equation}
    Per un cambio nel percorso $y_n \to y_n + \delta$ abbiamo un $\Delta\action_E$ di
    \begin{equation}
        \Delta\action_E = \sqrt{d} \sum_{n=0}^N \frac{2y_n\delta + \delta^2}{\eta}  - \frac{y_{n-1} + y_{n+1}}{\eta}\delta + \eta \left( \tilde{V} \left(y_n + \delta\right) - \tilde{V} \left(y_n\right) \right)
    \end{equation}
    
    
\section{Validità dei limiti}

    I limiti di bassa temperatura e continuo sono ora $\beta \to +\infty$ e $\eta \omega_B \to 0$
    
    Il limite di bassa temperatura è raggiunto quando l'energia media della particella è situata nella parte quadratica della prima banda di Bloch. Essendo, come visto nell'approssimazione di \textit{tight binding} le bande centrate sui livelli della singola buca, una stima grossolana può essere $k_B T \ll \hbar \omega_B$. Il vincolo da imporre è quindi $\tilde{\beta} \gg 1$. Comunque un'ulteriore controllo si può fare osservando i cammini: essi devono visitare molteplici buche.
    
    $\eta$ deve essere molto inferiore alla più piccola scala temporale presente, ovvero quella di buca singola. Essendo già $\eta$ in unità di $\omega_B^{-1}$ il vincolo è $\eta \ll 1$.
    
\section{Misurare $\meff$}

    Nel tempo euclideo il meccanismo per cui $\meff$ emerge è chiaramente visibile: all'approfondirsi e/o allargarsi delle buche il cammino tipico è sempre più vincolato a restare in una sola di esse, e tende quindi a variare meno rispetto a quello di una particella libera.
    Effettivamente la distanza quadratica media percorsa dalla particella in un tempo $\tau$ può essere risolta esattamente nel caso di particella libera con massa $\meff$:
    \begin{equation}\label{eq:free_mean_path}
        \left\langle\left(x(\tau) - x(0)\right)^2\right\rangle = \frac{4}{\meff \beta} \sum_{n=1}^{+\infty} \frac{1 - \cos\left(\omega_n \tau\right)}{\omega_n^2} \quad
        \omega_n = 2\pi\frac{n}{\beta \hbar}
    \end{equation}
    Ci aspettiamo che per $\beta \hbar$ abbastanza grande (basse temperature) anche la particella soggetta a un potenziale periogico segua una distibuzione simile, composta però da un \textit{random walk} di instantoni tra una buca e l'altra. Per le alte temperature invece la particella "non fa in tempo" a colpire le pareti della buca, apparendo effettivamente libera.
    Dato che l'espressione \eqref{eq:free_mean_path} dipende da $\meff$ solo per un fattore, eseguire un fit non è utile alla bontà della misura: possiamo quindi mediare su $\tau$ e sfruttare l'identità $\zeta(2)=\frac{\pi^2}{6}$ per ottenere l'osservabile finale:
    \begin{equation}\label{eq:observable}
       \left\langle \frac{1}{\beta \hbar} \int \diff\tau \left(x(\tau) - x(0)\right)^2\right\rangle = \frac{4}{\meff \beta} \left(\frac{\beta \hbar}{2 \pi}\right)^2 \sum_{n=1}^{+\infty} \frac{1}{n^2} = \frac{\beta \hbar^2}{6 \meff}
    \end{equation}
    Usando le sostituzioni \eqref{eq:adimensional_subs} possiamo adimensionalizzare:
    \begin{equation}
       \left\langle \frac{1}{\tilde{\beta}} \int \diff z \left(y(z) - y(0)\right)^2\right\rangle = \frac{\tilde{\beta}}{6 \sqrt{d}} \frac{m}{\meff}
    \end{equation}
    Infine la media su $z$ va discretizzata. Possiamo rendere più esatta il calcolo per la particella libera: la discretizzazione ferma la somma alla frequenza di Nyquist $\omega_{\frac{N}{2}}$. Inserendo il fattore $f(N)=\frac{6}{\pi^2}\sum_{n=1}^{\frac{N}{2}} n^{-2}$ la stima finale è
    \begin{equation}
        \frac{\meff}{m} = \frac{\eta N f(N)}{6 \sqrt{d}} \left(
        \left\langle \frac{1}{N} \sum_{n=0}^{N-1}  \left(y_n - y_0\right)^2\right\rangle 
        \right)^{-1}
    \end{equation}
    

\section{Test della simulazione}
    
    Il rapporto $\frac{m_{eff}}{m}$ dipende dalla funzione $\tilde{V}(y)$ e dal parametro di dimensione delle buche adimensionale $d = \frac{V_0 l^2 m}{\hbar^2}$
    Il caso speciale di potenziale a onda quadra gode di una soluzione esatta per la funzione d'onda. Purtroppo non ho trovato una soluzione analitica della dipendenza di $m_{eff}$ da $d$, ma è possibile uno studio numerico a partire dalla condizione per le bande di Bloch.
    Per $d < 100$, un'approssimazione quadratica è ottima e otteniamo:
    \begin{equation}
        \frac{m_{eff}}{m} = 1 + \left(0.00589 \pm 3.2 \times 10^{-8}\right)\left(\frac{V_0 l^2 m}{\hbar^2}\right)^2
    \end{equation}
    Voglio quindi confermare questo risultato, e trovare i coefficienti per diversi $\tilde{V}(y)$
    
\end{document}